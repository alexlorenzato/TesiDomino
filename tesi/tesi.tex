% !TEX root =path to root.tex
\documentclass[a4paper.12pt]{report} % "report" è ottimo per tesi, include chapter

\usepackage[utf8]{inputenc}          % Supporta caratteri con accenti
\usepackage[italian]{babel}
\usepackage{amsmath, amssymb}         % Pacchetti per la matematica avanzata
\usepackage{graphicx}                 % Per inserire immagini
\usepackage{hyperref}                 % Per link interni
\usepackage{geometry}                 % Imposta margini del documento
\geometry{left=3cm, right=3cm, top=3cm, bottom=3cm}

\begin{document}

\title{Studio sul Gioco del Domino}
\author{Alex Lorenzato}
\maketitle

\tableofcontents % Crea automaticamente un indice dei contenuti

%TODO: capitolo che parla del gioco, inserire un pò di storia del gioco, parlare in maniera approfondita della variante che stiamo considerando, citare altre varianti del gioco, noi stiamo considerando la blocks, esiste anche la variante draw, queste sono le due principali, pooi ce ne sono altre
%? devo usare la prima persona singolare?


\chapter{Introduzione}

\section{Il Domino}

Il domino è un antico gioco da tavolo, le cui origini risalgono alla Cina imperiale. Le tessere di gioco sono rettangolari e suddivise in due metà; su ciascuna metà è presente un valore numerico che varia tra 0 e 6, rappresentato tipicamente con dei pallini, o "pip", simili a quelli presenti sui dadi che indicano il punteggio della tessera. Le tessere formano una combinazione unica di numeri, con 28 tessere totali in un set "double-six", ovvero la cui tessere di maggior valore è quella in cui appaiono due 6.

L'obiettivo generale del gioco, sebbene esistano diverse varianti, è di liberarsi di quante più tessere possibili dalla propria mano prima della fine della partita. Il gioco procede a turni in senso orario: ogni giocatore, al suo turno, ha la possibilità di aggiungere una tessera dalla propria mano all'inizio o alla fine della sequenza di tessere presenti sul tavolo, rispettando la condizione che il valore su una delle estremità della tessera coincida con il valore della tessera a cui si collega. 

Nella variante classica, la partita inizia con il giocatore che possiede la tessera doppia con il valore più alto, di conseguenza il doppio-6 è la tessera che garantisce di partire per primi. Questa versione è particolarmente popolare nella modalità a 2 giocatori, ma può essere giocata anche da un massimo di 4 o 6 giocatori, con regole e adattamenti per gestire il numero di tessere e i turni.

\section{Storia del Domino}

Il domino con tessere nacque in Cina intorno al XII secolo e si ritiene che sia stato sviluppato da uno statista nel 1120, come dono per l'imperatore Hui Tsung. Inizialmente, il domino aveva anche un ruolo come strumento di divinazione intorno al XIII secolo, diventando poi un gioco da tavolo vero e proprio.

Nel corso del tempo, il gioco del domino si diffuse dal continente asiatico all'Europa, probabilmente grazie agli scambi culturali mediati dal mondo arabo. Giunto in Italia, il gioco si diffuse rapidamente in Francia e successivamente in tutta Europa. Il termine "domino" deriva dal latino \textit{dominus}, che significa "padrone", e sembra essere legato al senso di controllo e strategia che il gioco richiede.\footnote{Fonte: Wikipedia, "Domino", https://it.wikipedia.org/wiki/Domino (accesso il 25 ottobre 2024).}

\section{Variante "Block"}

La variante "Block" è la versione del domino utilizzata come oggetto di studio in questa tesi, nonché la variante più diffusa e popolare.

Questa versione utilizza un set di tessere "double-six", ovvero tessere con valori che vanno da 0 a 6 su ciascuna metà. Ogni giocatore pesca sette tessere all'inizio del gioco, e le rimanenti 14 tessere vengono scartate e non utilizzate per tutta la durata della partita. La partita termina in due casi:

\begin{enumerate}
    \item Quando un giocatore riesce a liberarsi di tutte le tessere in mano, vincendo automaticamente.
    \item Quando la partita è "bloccata", ossia nessuno dei due giocatori può posizionare una tessera valida.
\end{enumerate}

In caso di "blocco", vince il giocatore con il minor punteggio totale in mano, dove il punteggio è calcolato come la somma dei valori presenti sulle tessere rimanenti. Se un giocatore riesce a esaurire tutte le sue tessere, è dichiarato automaticamente vincitore.


\section{Altre varianti}

Assieme alla variante "Block", esiste la variante "Draw", che differisce per l'uso delle tessere non distribuite ai giocatori. Mentre nella "Block" le tessere avanzate vengono rimosse definitivamente dalla partita, nella variante "Draw" i giocatori possono pescare tessere aggiuntive quando non riescono a fare una mossa valida. Le modalità di pesca variano: in alcune versioni il giocatore è obbligato a pescare fino a trovare una tessera giocabile, mentre in altre può pescare solo una tessera per turno.


Queste varianti danno origine a una vasta gamma di regole aggiuntive e variazioni del gioco. Esistono inoltre numerose versioni del domino fan-made e competitive, ognuna con modifiche più o meno significative alle regole delle varianti "Block" e "Draw". Per esempio, oltre all’obiettivo tradizionale di esaurire la propria mano, alcune modalità premiano configurazioni particolari di tessere sul tavolo, rendendo il gioco ancora più strategico e vario.


\chapter{Risultati}

\section{Statistiche}

Il numero totale possibile di configurazioni delle mani iniziali nel domino è: \(137.281.098.240\), ed è calcolato nel seguente modo:


\begin{enumerate}
    \item Possibili set da 14 tessere: \(\binom{28}{14} = 40.116.600\), ai quali bisogna sottrarre tutti i set che non contengono nessuna tessera doppia.
    \item Set che non contengono nessuna tessera doppia: \(\binom{21}{14} = 116.280\).
    \item Possibili set da 14 tessere validi per poter iniziare una partita: \(40.116.600 - 116.280 = 40.000.320\), ovvero il totale di set possibili esclusi quelli senza una tessera doppia.
    \item Ognuna delle \(40.000.320\) possibili partite è costituita da un set di 14 tile, che però possono essere distrubuite in \(\binom{14}{7} = 3432\) modi diversi tra i 2 giocatori
    \item Il numero totale di partite esistenti con un set da 28 tile e mani da 7 tile ciascuna quindi è: \(40.000.320 * 3432 = 137.281.098.240\)
\end{enumerate}

Testare l'algoritmo su tutte le possibili partite non è temporalmente possibile con gli strumenti a disposizione, questo perché supponendo una media di 70ms a partita (media calcolata sui dati raccolti) ci vorrebbero \(137.281.098.240 * 70  = 9.609.676.876.800\) ms, equivalenti a \(304.7\) anni su un sistema monoprocessore; avendo avuto a disposizione un server con 40 processori, il tempo di calcolo necessario sarebbe stato \(304 / 40 \tilde= 7.5\) anni. 

Altre statistiche interessanti: 
\begin{enumerate}
    \item ci sono \( \) partite con una sola foglia, equivalenti al \( X\%\), questo accade perché\dots
    \item 
\end{enumerate}

Una sola foglia vuol dire che la partita ha una sola mossa possibile.. spiega
il numero di foglie determina la complessità della partita

Partita con numero di foglie massimo:  e ha mani: .. questo accade perché.. spiega

altra cosa interessante: se le partite con 1 foglia sono banali, ancora più banali sono quelle con più foglie ma il risultato non cambia, quindi indipententemente dalla bravura l'esito è lo stesso, la % di queste è: 

Grafico della distribuzione del numero di foglie:



\chapter{Conclusioni}

Il set più scarso possibile è: .. questo perché

Il set più forte possibile è: .., questo perché



\appendix
\chapter{Codice sorgente}
% Includi eventuali porzioni di codice

\bibliographystyle{plain} % Stile della bibliografia
\bibliography{bibliografia} % File .bib con riferimenti

\end{document}
